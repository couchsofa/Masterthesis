%Zusammenfassung

In dieser Arbeit wird ein analytisches Modell erstellt mit dem isolierte Antwortspecktren erzeugt werden können mithilfe derer Gebäudeberechnungen durchgeführt werden können ohne der Isolator im Gebäudemodell erfasst werden muss.
Die isolierten Antwortspektren werden mithilfe des hier erarbeiteten Modells aus den Antwortspektren gemäß Eurocode 8 abgeleitet.
Dies hat den Vorteil, dass die Antwortspecktren nach bereits etablierten Verfahren für den individuellen Gebäudestandort ermittelt werden können.
Das Modell erspart dem Anwender ebenfalls die eventuell komplexe Modellierung des Gebäudeisolators, da das Verhalten des Isolators im isolierten Antwortspektrum abgebildet ist.
Es ist folglich möglich Verfahren wie die Modalanalyse oder das vereinfachte Antwortspektrum unter Verwendung des isolierten Antwortspektrums durchzuführen.

Um das Verhalten eines Isolatortyps im Modell abzubilden wurden in dieser Arbeit ausschließlich Gleitpendelisolatoren betrachtet.
Gute Isolationseigenschaften werden erzielt, wenn die Steifigkeit des Isolators deutlich kleiner ist als die der aufgehenden Struktur und die Masse über dem Isolator möglichst groß ist.
Die Rückstell- und Reibungskrtaft ist bei Gleitpendelisolatoren proportional zu ihrer Steifigkeit.
Die Dämfung wird in linearer Näherung über die Hystereseschleife der steifigkeit berechnet.

Um die Ergebnisse des isolierten Antwortspecktrums abschätzen zu können werden zunächst näherungen mithilfe des effektiven Einmassenschwingers berechnet. \cite{Kelly}
Das isolierte Antwortspektrum wird dabei zunächst mit einem vereinfachten Verfahren berechnet. Dieses Verfahren liefert jedoch Werte auf der unsicheren Seite, wesshalb ein weiterer Ansatz untersucht wird.
Dieser zweite Ansatz basiert auf der Berechnung der Transmissibilität um die Übertragung von Schwingungen aus Fußpunktanregungen auf die aufgehende Struktur via des Isolators zu bestimmen.
Bei diesem  Verfahren ist zudem möglich unterschiedliche Dämpfungswerte für Struktur und Isolator anzunehmen.
Im Verlauf dieser Arbeit wurden daher verschiedene Dämpfungsmodelle untersucht.

Die Korrektheit des Ableitungsverfahrens und der Dämpfungsmodelle wurde nachfolgend durch vergleichende Beispielrechznungen verifiziert.
Die Ergebnisse wurden mit den Ergebnissen einer numerischen Zeitschrittberechnung und denen des vereinfachten Verfahrens verglichen.
Bei den Vergleichen zeigten sich, dass das Transmisibilitätsverfahren Werte erzeugt die auf der sicheren seite liegen und den Qualitativen Verlauf aus den Zeitschrittberechnungen besser abbildet.

Mit einer Modalanalyse wurde weiterhin gezeigt, dass das isolierte Antwortspektrum dazu verwendet werden kann das Verhalten des Isolators abzubilden, ohne das dieser in dem Gebäudemodell vorhanden sein muss.
Es wurden so statische Ersatzlasten berechnet, mithilfe derer das Gebäude auf Erdbebensicherheit ausgelegt werden kann, ohne das der Isolator aufwändig in das Gebäudemodell integriert werden muss.


\section{Aussicht}

Die maximale auslenkung muss noch berechnet werden
Korrekturwert in Abhängigkeit von R finden für D
Korrekturwert in Abhängigkeit der Isolatorparameter  für eta finden

\section{Excel Sheet zur Berechnung von Isolationsspektren}

Excel Sheet
	Excel Sheet nach zweitem Ansatz

\pagebreak