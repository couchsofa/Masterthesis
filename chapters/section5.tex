Besonders interessant wäre es gewesen, die Ergebnisse einer Modalanalyse mit denen aus einer Zeitschrittberechnung zu vergleichen. Leider konnte aber das Modell mit den Angaben aus Kapitel 13 \cite{Isemann} nicht reproduziert werden.
Das grundlegende Vorgehen und die Ergebnisse wurden dort aber bereits erläutert und diskutiert. An dieser Stelle soll aber dennoch, wenn auch an einem Modell mit anderen Parametern, gezeigt werden, dass eine Modalanalyse möglich ist und wie diese mit einem Isolationsspektrum durchgeführt werden kann, um letztendlich Ersatzlasten für den Erdbebenlastfall zu generieren.

Daher wird hier eine Modalanalyse mit zwei Isolationsspektren in dem Programm \emph{RStab} durchgeführt und die Ergebnisse verglichen. Denn das letztendliche Ziel soll es sein, die erzeugten Isolationsspektren in einer computergestützten Modalanalyse zur Vordimensionierung zu verwenden.

\pagebreak

\section{Beispielgebäude}
\label{sec:besipielgebaude}

Das Gebäude wurde als Mehrmassenschwinger vereinfacht modelliert. Der Fußpunkt wird dabei fest eingespannt und die Knoten der Stockwerke als nur seitlich verschieblich modelliert. Die Steifigkeiten der Stäbe betragen je

\begin{align*}
EI_z &= EI_y = 32 kN/m^2 \cdot 1238690.27 m^4\\
     &= 39638088.6 kNm^2
\end{align*}

und die Höhen und Massen der Stockwerke

\begin{table}[H]
\centering
\begin{tabular}{ |c|c|c| } 
 \hline
 - & Masse [t] & Höhe [m]\\
 \hline\hline
4. OG & 466.0 & 3.20\\
3. OG & 505.2 & 3.30\\
2. OG & 505.2 & 3.30\\
1. OG & 505.2 & 3.30\\
EG    & 505.2 & 3.30\\
 \hline \hline
$\Sigma$ & 2456.8 & 16.4\\
\hline
\end{tabular}
\end{table}

Die Parameter des Isolators zur Erzeugung der Isolationsspektren wurden wie folgt gewählt.

\makebox[1cm]{$D$}    = 0.325 m \par
\makebox[1cm]{$\mu$}  = 0.05\par
\makebox[1cm]{$m_1$}  = 2486.7 t\par
\makebox[1cm]{$m_2$}  = 1619.5 t\par
\makebox[1cm]{$k_2$}  = 32000 kN/m \par
\makebox[1cm]{$R$}    = 1.599 m\par

Daraus ergeben sich die Antwortspektren wie in \cref{fig:Isolation2}.

\pagebreak

\section{Berechnung mit RStab}
\label{sec:rstab}

In \emph{RStab} wurden die Perioden und Beschleunigungen direkt eingegeben, um ein benutzerdefiniertes Antwortspektrum zu erzeugen.

\begin{table}[H]
\centering
\begin{tabular}{ |c|c|c| } 
 \hline
 $T [s]$ & $S_a$ Transmissibilität $[m/s^2]$ & $S_a$ Vereinfacht $[m/s^2]$\\
 \hline\hline
0.01 & 5.6067 & 5.124\\
0.20 & 5.6346 & 5.083\\
0.40 & 5.7126 & 5.010\\
0.60 & 5.8303 & 4.945\\
0.80 & 5.9736 & 4.839\\
1.00 & 6.1238 & 4.714\\
1.20 & 6.2564 & 4.550\\
1.40 & 6.3435 & 4.396\\
1.60 & 6.3582 & 4.200\\
1.80 & 6.2816 & 3.972\\
2.00 & 6.1083 & 3.761\\
2.20 & 5.8472 & 3.475\\
2.40 & 5.5186 & 3.216\\
2.60 & 5.1478 & 2.894\\
2.80 & 4.7590 & 2.550\\
3.00 & 4.3719 & 2.280\\
3.20 & 4.0003 & 2.110\\
3.40 & 3.6527 & 1.892\\
3.60 & 3.3332 & 1.701\\
3.80 & 3.0429 & 1.457\\
4.00 & 2.7812 & 1.301\\
 \hline
\end{tabular}
\caption{Spektralbeschleunigungen der Isolationsspektren}
\end{table}

\pagebreak

\begin{figure}[H]
    \centering
    \includegraphics[width=1.0\textwidth]{RSTAB_AWS_1.png}
    \caption{Isolationsspektrum nach Ansatz der Transmissibilität in \emph{RStab}}
\end{figure}

\begin{figure}[H]
    \centering
    \includegraphics[width=1.0\textwidth]{RSTAB_AWS_2.png}
    \caption{Isolationsspektrum nach vereinfachtem Ansatz in \emph{RStab}}
\end{figure}

Mit dem Zusatzmodul \emph{DYNAM Pro} wurden dann Ersatzlasten in Höhe der Decken über den jeweiligen Geschossen generiert.

\begin{table}[H]
\centering
\begin{tabular}{ |c|r|r| } 
 \hline
 $-$ & $S_a$ Transmissibilität $[kN]$ & $S_a$ Vereinfacht $[kN]$\\
 \hline\hline
4. OG & 4039.2 & 3007.0\\
3. OG & 3175.4 & 2363.9\\
2. OG & 1998.2 & 1487.6\\
1. OG &  986.8 &  734.6\\
EG    &  272.5 &  202.6\\
 \hline
\end{tabular}
\caption{Mit \emph{DYNAM Pro} generierte Ersatzlasten}
\end{table}

\pagebreak