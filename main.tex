% Gobals
\newcommand{\GetAuthor}{Arne Rick}
\newcommand{\GetTitle}{Erzeugung Modifizierter Antwortspektren zur Vordimensionierung von Seismisch Isolierten Bauwerken}

% Layout
\documentclass[12pt, oneside]{report}
\usepackage[a4paper,width=150mm,top=25mm,bottom=25mm,bindingoffset=6mm,headheight=28pt]{geometry}
\usepackage[utf8]{inputenc}

\usepackage[T1]{fontenc}
\usepackage{parskip}
\usepackage[ngerman]{babel}

\usepackage{graphicx}
\graphicspath{ {images/} }

\usepackage{pdfpages}

%\usepackage{listings}
%\usepackage{solarized-light}
%\usepackage{xcolor-solarized}

%\lstset{numbers=left,
%		numbersep=0.5em,
%		numberstyle=\normalfont\footnotesize\color{solarized-violet},
%		frame=l,
%		framesep=2em,
%		framerule=1pt,
%		fillcolor=\color{solarized-base2}, 
%	    rulecolor=\color{solarized-base2},
%	    rulesepcolor=\color{solarized-base2},
%	    xleftmargin=2em}

%\newenvironment{inconsolata}
%	{\fontfamily{fi4}\footnotesize\selectfont}
%	{\par}

\usepackage{multicol}

\usepackage{longtable}

\usepackage{wrapfig}
\usepackage{subfig}
\usepackage{float}
\floatstyle{boxed} 
\restylefloat{figure}

\usepackage{amsmath}
\usepackage{rotating}

\usepackage{tikz}
\usetikzlibrary{calc}

\setcounter{tocdepth}{4}
\setcounter{secnumdepth}{4}

\usepackage{scrextend}

\usepackage{csvsimple}


% Notes
\usepackage[colorinlistoftodos,prependcaption,textsize=tiny]{todonotes}
\newcommand{\note}[1]{\todo[linecolor=red,backgroundcolor=red!25,bordercolor=red]{#1}}

% Header and footer
\usepackage{fancyhdr}
\pagestyle{fancy}

\lhead{}
\chead{\GetTitle{}}
\rhead{}

\lfoot{Kapitel \thechapter}
\cfoot{\GetAuthor{}}
\rfoot{\thepage}

%\fancyhead{}
%\fancyhead[CO,LE]{\GetTitle{}}
%\fancyfoot{}
%\fancyfoot[LE,RO]{\thepage}
%\fancyfoot[LO,CE]{Chapter \thechapter}
%fancyfoot[CO,RE]{\GetAuthor{}}
\renewcommand{\headrulewidth}{0.4pt}
\renewcommand{\footrulewidth}{0.4pt}

% References
\usepackage{hyperref}
\usepackage[style=numeric]{biblatex}
\addbibresource{references.bib}

\usepackage{cleveref}
\Crefname{figure}{Abbildung}{Abbildungen}
\Crefname{figure}{Abbildung}{Abbildungen}
\Crefname{equation}{Gleichung}{Gleichungen}
\Crefname{chapter}{Kapitel}{Kapitel}
\Crefname{section}{Abschnitt}{Abschnitte}
\Crefname{subsection}{Abschnitt}{Abschnitte}
\Crefname{subsubsection}{Abschnitt}{Abschnitte}

\crefname{figure}{Abbildung}{Abbildungen}
\crefname{figure}{Abbildung}{Abbildungen}
\crefname{equation}{Gleichung}{Gleichungen}
\crefname{chapter}{Kapitel}{Kapitel}
\crefname{section}{Abschnitt}{Abschnitte}
\crefname{subsection}{Abschnitt}{Abschnitte}
\crefname{subsubsection}{Abschnitt}{Abschnitte}

% Title
\title{
	{\GetTitle{}}\\
	{\includegraphics{hska_logo.png}}
}
\author{\GetAuthor{}}
\date{\today{}, Karlsruhe}

% Document
\begin{document}

% Notes
%\newpage
%\listoftodos[Notes]
%\newpage

\maketitle
\setcounter{page}{1}

\chapter*{\centering Abstract}
In dieser Arbeit wurden verschiedene Ansätze zur Erzeugung von modifizierten Antwortspektren für seismisch isolierte Bauwerke untersucht und versucht einen analytischen Ansatz zu finden, mit dem es möglich ist diese aus den Antwortspektren des Eurocode 8 zu generieren.
Mit diesen Isolationsspektren kann für jede Periode der Struktur über dem Isolator eine Spektralbeschleunigung ermittelt und eine Modalanalyse durchgeführt werden.
Ein Bauwerk kann dann getrennt von dem Isolator betrachtet und zwecks einer Vordimensionierung mit einem Gebäudemodell auf die Erfüllung der Anforderungen an die Erdbebensicherheit untersucht werden.
Der analytische Ansatz wird in einem Tabellenkalkulationsprogramm implementiert um Isolationsspektren in Abhängigkeit der Bauwerks- und Isolatorparameter zu erzeugen.

\chapter*{\centering Danksagung}

Mein Dank gilt vorallem Prof. Dr.-Ing. Jan Akkermann für die Idee dieser Arbeit und seine stetige Beratung bei der Bearbeitung.
Auch meinen Freunden Jan und Alexander sei an dieser Stelle Dank für die inspierenden Gespräche zu dem Thema ausgesprochen.

\pagestyle{empty}

\cleardoublepage
\tableofcontents
\cleardoublepage
\addtocontents{toc}{\protect\thispagestyle{empty}}

\clearpage
\pagestyle{fancy}

\chapter{Einleitung}
\section{Erdbeben}
\label{sec:erdbeben}

\pagebreak

\section{Berechnung}
\label{sec:berechnung}

\pagebreak

\section{Vereinfachte Verfahren}
\label{sec:vereinfachteverfahren}

\pagebreak

\section{Vordimensionierung}
\label{sec:vordimensionierung}

\pagebreak

\chapter{Isolatoren}
Wenn die Einwirkungen aus Erdbeben sehr hoch werden, zum Beispiel durch eine hohe Anforderung an den Bedeutungsbeiwert oder wenn sich das Bauwerk in einem Starkbebengebiet befindet, ist es meistens technisch und wirtschaftlich günstiger, die Struktur von der Einwirkung zu isolieren, damit sie dieser nicht mehr vollends ausgesetzt wird.

Es können leichtere Konstruktionen gebaut werden, die durch geringere Aufwendung an Material die Kosten senken und die Nachhaltigkeit durch Senken des Ausstoßes an CO\textsubscript{2} erhöhen.

Die horizontale Isolation ist keine Lösung der Neuzeit. Schon die Baumeister im alten China ordneten zwischen Fundament und Grundplatte eine Schicht aus rolligem Sand an. \cite{Taylor}
Im 20. Jahrhundert folgten einige Patente mit dem selben Grundprinzip und 1921 realisierte Frank Lloyd Wright das Imperial Hotel in Tokyo mit einer Isolation mittels einer 3m mächtigen Schicht aus Weichboden. Das Gebäude überstand ein zwei Jahre später aufgetretenes schweres Erdbeben nahezu unbeschadet. \cite{Reitherman}

Für die Isolierung stehen einige verschiedene Mechanismen, wie zum Beispiel kinematische Lager, Gleitpendelisolatoren und Elastomerlager (ggf. mit Bleikern) zur Verfügung.
In dieser Arbeit sollen nur Gleitpendelisolatoren betrachtet werden.

\pagebreak

\section{Gleitpendelisolatoren}
\label{sec:gleitisolatoren}

 Gleitpendelisolatoren bestehen aus zwei spherisch angeformten Lagerplatten, zwischen denen ein Gleitschuh geschaltet wird.

\begin{figure}[H]
    \centering
    \includegraphics[width=0.9\textwidth]{maurer.png}
    \caption{Gleitpendelisolator [Maurer SE (maurer.eu)]}
\end{figure}

Die Reibung zwischen den Schnittstellen und somit die Energiedissipation kann eingestellt werden.
Bei einem zu hohen Reibkoeffizienten besteht jedoch die Gefahr, dass die Rückzentrierung nicht mehr gewährleistet werden kann, welche ein großer Vorteil der Gleitpendelisolatoren ist.
Zur Erhöhung der Dissipation können aber auch zusätzliche viskose Dämpfungselemente angeordnet werden (\cref{Dampener}).

\begin{figure}[H]
    \centering
    \subfloat[Gleitpendelisolator]{{\includegraphics[width=0.45\linewidth]{Algerien_1.png} }}%
    \qquad
    \subfloat[Viskoser Hydraulikdämpfer]{{\includegraphics[width=0.45\linewidth]{Algerien_2.png} }}%
    \caption{Bauform der Iolatoren (a) und Dämpfer (b) der Großen Moschee von Algerien \cite{AKK}}%
\end{figure}

\begin{figure}[H]
    \centering
    \includegraphics[width=0.9\textwidth]{Algerien_3.png}
    \caption{Verteilung der Isolatoren (grün) und Dämpfer (rot) im Grundriss \cite{AKK}}
	\label{Dampener}
\end{figure}

\pagebreak

\section{Funktionsweise}
\label{sec:funktion}

Isolatoren stellen eine Ebene zwischen der Gründung und dem aufgehendem Bauwerk dar. Sie haben eine deutlich geringere Steifigkeit als die zu isolierende Struktur, wodurch zwar große Verschiebungen am Isolator auftreten (\cref{Verteilung}), aber die Grundschwingzeit reduziert wird.
Die relativen Verschiebungen der Struktur werden verringert und somit die Beschleunigungen und ebenso die Trägheitskräfte der Massen reduziert.

\begin{figure}[h]
    \centering
    \includegraphics[width=0.9\textwidth]{Verschiebung_iso.png}
    \caption{Verteilung der Verschiebungen an einem isolierten System \cite{Kelly}}
	\label{Verteilung}
\end{figure}

Die Dissipationsfähigkeit, Steifigkeit und Eigenfrequenz dieser Isolatoren kann über den Reibkoeffizienten, den Radius des Pendels und der Masse über dem Isolator beeinflusst werden.

\subsection{Abstimmung}
\label{sec:abstimmung}

Damit der Isolator möglichst effektiv wirkt, sollte das Ziel sein, die Masse direkt über dem Isolator möglichst groß zu wählen und die Steifigkeit zu reduzieren, wobei die aufgehende Struktur möglichst steif sein sollte.
Dadurch sollte die Periode des Isolators $T_I$ möglichst weit von der Periode der Struktur $T_S$ entfernt sein.
Ein Wert, der sich in der Praxis als anstrenswert erwiesen hat, ist $T_I \approx 3 \cdot T_S$.

\subsection{Steifigkeit}
\label{sec:steifigkeit}

Die effektive Steifigkeit kann über die Rückstell- und Reibkraft des Gleitpendellagers berechnet werden. Die Rückstellkraft wird durch die Anhebung der Vertikalkraft (Eigengewicht des Bauwerks) ausgelöst. \cite{Pocanschi}

\begin{figure}[H]
    \centering
    \includegraphics[width=0.9\textwidth]{Pendellager.png}
    \caption{Schematischer Aufbau des Gleitpendellagers im zentriertem sowie im ausgelenkten Zustand \cite{Romen}}
\end{figure}

\begin{align*}
F_{\text{Rück}} &= \frac{G D}{R \cos \theta}\\
F_{\text{Reib}} &= \mu G
\end{align*}

\makebox[0.8cm]{$G$}  Vertikalkraft (Eigengewicht)\par
\makebox[0.8cm]{$D$}  Auslenkung\par
\makebox[0.8cm]{$R$}  Radius der Isolator-Gleitfläche\par
\makebox[0.8cm]{$\theta$}  Winkel der Auslenkung\par
\makebox[0.8cm]{$\mu$}  Reibungskoeffizient des Isolators\par

Für kleine Winkel mit $\cos \theta = 1$ ergibt sich die Steifigkeit zu:

\begin{align}
k_{eff} &= \frac{F_{\text{Rück}} + F_{\text{Reib}}}{D}\nonumber\\
        &= \frac{G}{R} + \mu \frac{G}{D}\label{keff}
\end{align}

\subsection{Dämpfung}
\label{sec:daempdung}

Die effektive Dämpfung ergibt sich aus der Fläche der Hystereseschleife und der effektiven Steifigkeit des Gleitpendellagers. \cite{Huber}\cite{Pocanschi}

\begin{figure}[h]
    \centering
    \includegraphics[width=0.9\textwidth]{Hysteresis.png}
    \caption{Hysterese-Zyklus [HDR Engineering Inc.]}
\end{figure}

\begin{align}
\xi_{eff} &= \frac{4 \mu G D}{2 \pi k_{eff} D^2}\nonumber\\
          &= \frac{2}{\pi} \frac{\mu R}{(D + \mu R)}\label{xieff}
\end{align}



\pagebreak

\section{Schwierigkeiten bei der Vordimensionierung}
\label{sec:schwierigkeitenvordimensionierung}

Für eine genaue Berechnung ist es sinnvoll, ein Gebäudemodell samt Isolator zu erstellen und mit dem Zeitschrittverfahren und Erdbebenzeitverläufen zu berechnen. Dies ist jedoch sehr aufwendig und bei einer Vordimensionierung nicht immer praktikabel, da sich Parameter noch ändern können.
Ein Ansatz ist es, die Gesamtstruktur auf einen Einmassenschwinger mit der effektiven Masse aus Struktur ($m_S$) und Isolator ($m_I$) zu vereinfachen. Er beruht auf der Annahme, dass die Steifigkeit der Struktur sehr hoch ist und die des Isolators $k_I$ die Eigenform somit dominiert. \cite{Kelly2}
Die Eigenfrequenz kann dann mit

\begin{equation}
\omega = \sqrt{\frac{k_I}{m_S + m_I}}
\end{equation}

bestimmt und die Spektralbeschleunigung $Sa(\frac{2 \pi}{\omega})$ aus dem Antwortspektrum entnommen werden.

Soll allerdings eine Modalanalyse am Gebäude mittels EDV vorgenommen werden, wird ein isoliertes Antwortspektrum benötigt.
So könnte man ein grobes Gebäudemodell erstellen und mit dem isolierten Antwortspektrum eine computergestützte Modalanalyse durchführen.

\pagebreak

\chapter{Berechnung des modifizierten Antwortspektrums}
\section{Modellierung}
\label{sec:modellierung}

\pagebreak

\section{Betrachtung als Übertragungsfunktion}
\label{sec:ubertragungsfunktion}

\pagebreak

\section{Vergleich der Systeme als Ein- und Zweimassenschwinger}
\label{sec:1ms2ms}

\pagebreak

\section{Grenzfälle}
\label{sec:grenzfalle}

\pagebreak

\section{Nichtlinearitäten und Ansätze zur Linearisierung}
\label{sec:nichtlinearitaten}

\pagebreak

\chapter{Vergleich der Ansätze am Beispiel}
Die Ansätze sollen anhand eines konkreten Beispiels verglichen werden. Hierzu werden die Werte aus \cite{Isemann} Kapitel 11.2.3 verwendet.
Die Parameter des Systems lauten:

\makebox[1cm]{$D$}    = 0.35 m \par
\makebox[1cm]{$\mu$}  = 0.04\par
\makebox[1cm]{$m_1$}  = 2486.7 t\par
\makebox[1cm]{$m_2$}  = 1619.5 t\par
\makebox[1cm]{$R$}    = 2.5 m\par

Es soll ein Antwortspektrum mit folgenden Eckperioden und Bodenbeschleunigung verwendet werden.

\makebox[1cm]{$T_B$}  = 0.4 s\par
\makebox[1cm]{$T_C$}  = 1.6 s\par
\makebox[1cm]{$T_D$}  = 2.0 s\par
\makebox[1cm]{$a_g$}  = 3.924 m/s\textsuperscript{2}\par

\begin{figure}[h]
    \centering
    \includegraphics[width=0.9\textwidth]{AWS_beispiel.png}
    \caption{Antwortspektrum (Beispiel 1)}
\end{figure}

Die Steifigkeit $k_2$ ergibt sich aus der effektiven Steifigkeit des Isolators nach \cref{keff}

\begin{equation*}
k_2 = k_{eff} = \frac{G}{R} + \mu \frac{G}{D} = \frac{41062 kN}{2.5 m} + 0.04 \cdot \frac{41062 kN}{0.35 m} = 21117 kN/m
\end{equation*}

Die Dämpfung $\xi_1$ wird mit 5 \% angenommen und die effektive Dämpfung des Isolators $\xi_2$ nach \cref{xieff} bestimmt.

\begin{equation*}
\xi_2 = \xi_{eff} = \frac{2}{\pi} \frac{\mu R}{(D + \mu R)} = \frac{2}{\pi} \frac{0.04 \cdot 2.5 m}{(0.35 m + 0.04 \cdot 2.5 m)} \approx 0.14147
\end{equation*}

\pagebreak

Hier sollen zunächst die Ergebnisse für eine Periode der aufgehenden Struktur von $T = 1 s$ betrachtet werden. Daraus ergibt sich eine Steifigkeit $k_1$ von

\begin{equation*}
k_1 = m_1 (\frac{2 \pi}{T})^2 = 2486.7 t \cdot (\frac{2 \pi}{1 s})^2 = 98170 kN/m
\end{equation*}

\pagebreak

\section{Betrachtung als effektiver Einmassenschwinger}

Wie in \cref{sec:schwierigkeitenvordimensionierung} erwähnt, kann das System unter der Annahme, dass der Isolator die Eigenform dominiert (\cref{Verteilung}), auch als vereinfachter Einmassenschwinger modelliert werden.

Die Eigenperiode beträgt dann

\begin{equation*}
\omega = \sqrt{\frac{k_2}{m_1 + m_2}} = \sqrt{\frac{21117 kN/m}{4106.2 t}} \approx 2.267 \frac{1}{s}
\end{equation*}

\begin{equation*}
T = \frac{2 \pi}{\omega} = \frac{2 \pi}{2.267 \frac{1}{s}} \approx 2.77 s
\end{equation*}

Für eine Periode von $T = 2.77 s$ im Bereich $T_D \leq T$ und unter der Annahme, dass auch die Dämpfung des Isolators dominiert ($\eta=\sqrt{10/(5+14.147)} \approx 0.723$), beträgt die Spektralbeschleunigung

\begin{align*}
S_a(T) &= a_g \eta 2.5 \frac{T_C T_D}{T^2}\\
       &= 3.924 m/s^2 \cdot 0.723 \cdot 2.5 \cdot \frac{1.6 s \cdot 2.0 s}{(2.77 s)^2}\\
       &= \underline{\underline{2.958 m/s^2}}
\end{align*}

\pagebreak

\section{Vereinfachtes Verfahren}

Im ersten Schritt wird die Eigenkreisfrequenz des nicht isolierten Bauwerks $\omega$ und die Verhältniswerte $\alpha$ und $\beta$ ermittelt.

\begin{align*}
\omega &= \sqrt{\frac{k_1}{m_1}} = \sqrt{\frac{98170 kN/m}{2486.7 t}} = 6.2832 \frac{1}{s}\\
\end{align*}

\begin{align*}
\alpha &= \frac{k_2}{k_1} = \frac{21117 kN/m}{98170 kN/m} = 0.2151 & \beta  &= \frac{m_2}{m_1} = \frac{1619.5 t}{2486.7 t} = 0.65\\
\end{align*}

Damit lassen sich die Eigenkreisfrequenzen und Perioden des isolierten Systems bestimmen.

\begin{align*}
\omega_{L,1}^2 &= \frac{1 + \alpha + \beta - \sqrt{(1 + \alpha + \beta)^2 - 4 \alpha \beta}}{2 \beta} \omega^2\\
               &= \frac{1 + 0.215 + 0.65 - \sqrt{(1 + 0.215 + 0.65)^2 - 4 \cdot 0.215 \cdot 0.65}}{2 \cdot 0.65} \cdot (6.2832 \frac{1}{s})^2\\
               &= 4.7504 \Rightarrow  \omega_{L,1} = 2.18 \frac{1}{s}\\
\end{align*}

\begin{align*}
T_{L,1} &= \frac{2 \pi}{\omega_{L,1}} = \frac{2 \pi}{2.18 \frac{1}{s}} = 2.882 s
\end{align*}

Für die Periode von $T = 2.77 s$ im Bereich $T_D \leq T$ und der Dämpfung des Isolators mit $\eta=\sqrt{10/(5+14.147)} \approx 0.723$ beträgt die Spektralbeschleunigung

\begin{align*}
S_a(T) &= a_g \eta 2.5 \frac{T_C T_D}{T^2}\\
       &= 3.924 m/s^2 \cdot 0.723 \cdot 2.5 \cdot \frac{1.6 s \cdot 2.0 s}{(2.882 s)^2}\\
       &= 2.733 m/s^2
\end{align*}

\pagebreak

Um dann die maximale Beschleunigung zu erhalten, wird noch der erste Eigenvektor und Beteiligungsfaktor berücksichtigt.

\begin{align*}
r_1 &= \frac{1 + \alpha - \beta + \sqrt{(1 + \alpha + \beta)^2 - 4 \alpha \beta}}{2} \\
    &= \frac{1 + 0.215 - 0.65 + \sqrt{(1 + 0.215 + 0.65)^2 - 4 \cdot 0.215 \cdot 0.65}}{2}\\
    &= 1.137
\end{align*}

\begin{align*}
\vec{\Phi}_1 &= \binom{1/r_1}{1} = \binom{1/1.137}{1} = \binom{0.8795}{1}
\end{align*}

\begin{equation*}
L_1 = \frac{\vec{\Phi}_1^T M \vec{I}}{\vec{\Phi}_1^T M \vec{\Phi}_1} = \frac{
\begin{pmatrix}
  0.8795 & 1
\end{pmatrix}
\begin{bmatrix}
  1619.5 t & 0 t\\
  0 t & 2486.7 t
\end{bmatrix}
\begin{pmatrix}
  1\\
  1
\end{pmatrix}
}{
\begin{pmatrix}
  0.8795 & 1
\end{pmatrix}
\begin{bmatrix}
  1619.5 t & 0 t\\
  0 t & 2486.7 t
\end{bmatrix}
\begin{pmatrix}
  0.8795 \\
  1
\end{pmatrix}}
= 1.0458
\end{equation*}

\begin{equation*}
\ddot U_{max} = \vec{\Phi}_1 L_1 S_a(T_{L,1}, \xi_{L,1}) = 1.0 \cdot 1.0458 \cdot 2.733 m/s^2 = \underline{\underline{2.858 m/s^2}}
\end{equation*}

\pagebreak

\section{Verfahren der Transmissibilität}

Zunächst werden die Dämpfungsbeiwerte mit dem Ansatz der Rayleigh-Dämpfung ermittelt. Dafür werden die Eigenkreisfrequenzen des ungedämpften Systems benötigt.

\begin{align*}
\omega_{1,2}^2 &= \frac{(k_2 + k_1) m_1 + k_1 m_2 \pm \sqrt{((k_2 + k_1) m_1 + k_1 m_2)^2 - 4 m_2 m_1 k_2 k_1}}{2 m_2 m_1}\\
               \intertext{mit $(k_2 + k_1) m_1 + k_1 m_2 = (21117  + 98170 ) \cdot 2486.7  + 98170  \cdot 1619.5 = 455568212.9 $ \newline und $4 m_2 m_1 k_2 k_1 = 4 \cdot 1619.5  \cdot 2486.7  \cdot 21117  \cdot 98170 = 3.3370719 \cdot 10^{14}$}
               &= \frac{ 455568212.9 \pm \sqrt{((21117  + 98170 ) \cdot 2486.7  + 98170  \cdot 1619.5 )^2 - 3.33.. \cdot 10^{14}}}{2 \cdot 1619.5  \cdot 2486.7 }\\
               &\Rightarrow \omega_1 \approx 2.179 \frac{1}{s}\\
               &\Rightarrow \omega_2 \approx 10.410 \frac{1}{s}
\end{align*}

Damit können die normierten Komponenten der Eigenvektoren bestimmt werden.

\begin{align*}
\varepsilon_1 &= \frac{k_2 + k_1 - m_2 \omega_1^2}{k_1} \\
              &= \frac{21117 kN/m + 98170 kN/m - 1619.5 t \cdot (2.179 \frac{1}{s})^2}{98170 kN/m}\\
              &= 1.136
\end{align*}
\begin{align*}
\varepsilon_2 &= \frac{k_2 + k_1 - m_2 \omega_2^2}{k_1} \\
              &= \frac{21117 kN/m + 98170 kN/m - 1619.5 t \cdot (10.410 \frac{1}{s})^2}{98170 kN/m}\\
              &= -0.572
\end{align*}

\begin{align*}
\varphi_{11} &= \sqrt{\frac{1}{1 + \varepsilon_1^2}} = \sqrt{\frac{1}{1 + 1.136^2}} \approx 0.660\\
\varphi_{21} &= \varepsilon_1 \varphi_{11} = 1.136 \cdot 0.660 \approx 0.750\\
\varphi_{12} &= \sqrt{\frac{1}{1 + \varepsilon_2^2}} = \sqrt{\frac{1}{1 + (-0.572)^2}} \approx 0.867\\
\varphi_{22} &= \varepsilon_2 \varphi_{12} = -0.572 \cdot 0.867 \approx -0.497
\end{align*}

\pagebreak

Die generalisierten Massen und Steifigkeiten sind dann

\begin{align*}
m_2^* &= \varphi_{11}^2 m_2 + \varphi_{21}^2 m_1 = 0.660^2 \cdot 1619.5 t + 0.750^2 \cdot 2486.7 t\\
      &= 2108.3 t\\[2em]
m_1^* &= \varphi_{12}^2 m_2 + \varphi_{22}^2 m_1 = 0.867^2 \cdot 1619.5 t + -0.497^2 \cdot 2486.7 t\\
      &= 1833.8 t\\[2em]
k_2^* &= \varphi_{11}^2 (k_2 + k_1) - 2 \varphi_{21} \varphi_{11} k_1 + \varphi_{21}^2 k_1\\
      &= 0.660^2 \cdot (21117 kN/m +  98170 kN/m) - 2 \cdot 0.750 \cdot 0.660 \cdot 98170 kN/m\\
      &\phantom{{}=1} + 0.750^2 \cdot  98170 kN/m\\
      &= 10013.4 kN/m\\[2em]
k_1^* &= \varphi_{12}^2 (k_2 + k_1) - 2 \varphi_{22} \varphi_{12} k_1 + \varphi_{22}^2 k_1\\
      &= 0.867^2 \cdot (21117 kN/m +  98170 kN/m) - 2 \cdot -0.497 \cdot 0.867 \cdot 98170 kN/m\\
      &\phantom{{}=1} + -0.497^2 \cdot 98170 kN/m\\
      &= 198759.0 kN/m
\end{align*}

und die Eigenkreisfrequenzen der zwei Einmassenschwinger betragen

\begin{align*}
\omega_1^* &= \sqrt{\frac{k_2^*}{m_2^*}} = \sqrt{\frac{10013.4 kN/m}{2108.3 t}}\\
           &= 2.179 \frac{1}{s}
\end{align*}
\begin{align*}
\omega_2^* &= \sqrt{\frac{k_1^*}{m_1^*}} = \sqrt{\frac{198759.0 kN/m}{1833.8 t}}\\
           &= 10.410 \frac{1}{s}
\end{align*}

\pagebreak

Damit ergeben sich die Dämpfungsbeiwerte der Rayleigh-Dämpfung in der Modalform.

\begin{align*}
\alpha &= \frac{2 \omega_1^* \omega_2^* (\xi_2 \omega_2^* - \xi_1 \omega_1^*)}{\omega_2^{*2} - \omega_1^{*2}}\\
       &= \frac{2 \cdot 2.179 \frac{1}{s} \cdot 10.410 \frac{1}{s} \cdot (0.14147 \cdot 10.410 \frac{1}{s} - 0.05 \cdot 2.179 \frac{1}{s})}{(10.410 \frac{1}{s})^2 - (2.179 \frac{1}{s})^2}\\
       &\approx 0.597\\[2em]
\beta  &= \frac{2 (\xi_1 \omega_2^* - \xi_2 \omega_1^*)}{\omega_2^{*2} - \omega_1^{*2}}\\
       &= \frac{2 \cdot (0.05 \cdot 10.410 \frac{1}{s} - 0.14147 \cdot 2.179 \frac{1}{s})}{(10.410 \frac{1}{s})^2 - (2.179 \frac{1}{s})^2}\\
       &\approx 0.004
\end{align*}

\begin{align*}
c_1^* &= \alpha m_1^* + \beta k_1^* = 0.597 \cdot 1833.8 t + 0.004 \cdot 198759.0 kN/m\\
      &\approx 1909.1\\
c_2^* &= \alpha m_2^* + \beta k_2^* = 0.597 \cdot 1833.8 t + 0.004 \cdot 10013.4 kN/m\\
      &\approx 1300.0
\end{align*}

Durch Rücktransformation in die Normalform werden die Komponenten der Dämpfungsmatrix $C$ erhalten.

\begin{align*}
C &= \vec{\Phi}^{-T} C^* \vec{\Phi}^{-1}\\
  &\Rightarrow c_1 = 867538.6\\
  &\Rightarrow c_2 = 334635.2
\end{align*}

Die Eigenfrequenz der ersten, durch den Isolator gesteuerten, Eigenform beträgt dann

\begin{align*}
\omega_{1d} &= \omega_1 \sqrt{1 - \xi_2^2} = 2.179 \frac{1}{s} \cdot \sqrt{1 - 0.14147^2}\\
            &\approx 2.157 \frac{1}{s}\\
T_1         &= \frac{2 \pi}{\omega_{1d}} = \frac{2 \pi}{2.157 \frac{1}{s}}\\
            &\approx 2.912 s
\end{align*}

\pagebreak

Mit der Periode lässt sich bereits $S_a$ aus dem Antwortspektrum (mit $\eta=1$) ermitteln.

\begin{align*}
S_a(T) &= a_g \eta 2.5 \frac{T_C T_D}{T^2}\\
       &= 3.924 m/s^2 \cdot 2.5 \cdot \frac{1.6 s \cdot 2.0 s}{(2.912 s)^2}\\
       &= 3.702 m/s^2
\end{align*}

Die Transmissibilität kann nun mit den Dämpfungsbeiwerten der Rayleigh-Dämpfung bestimmt werden. Dazu werden zuerst die drei komplexwertigen Transmissionskoeffizienten (mit $\omega = \omega_{1d}$) berechnet.

\begin{align*}
X_1 &= \frac{\omega^2 m_1^*}{k_1^* + i \omega c_1^*} = \frac{(2.157 \frac{1}{s})^2 \cdot 1833.8 t}{198759.0kN/m + i \cdot 2.157 \frac{1}{s} \cdot 867538.6}\\
    &\approx 0.0003 - 0.006i\\[2em]
X_2 &= \frac{\omega^2 m_2^*}{k_2^* + i \omega c_2^*} = \frac{(2.157 \frac{1}{s})^2 \cdot 2108.3 t}{10019.4kN/m + i \cdot 2.157 \frac{1}{s} \cdot 334635.2}\\
    &\approx 0.0003 - 0.010i\\[2em]
X_{12} &= \frac{\omega^2 m_1^*}{k_2^* + i \omega c_2^*} = \frac{(2.157 \frac{1}{s})^2 \cdot 1833.8 t}{10019.4kN/m + i \cdot 2.157 \frac{1}{s} \cdot 334635.2}\\
    &\approx 0.0004 - 0.016i
\end{align*}

\begin{align*}
VT &= \left\lvert \frac{1}{(1 - X_1)(1 - X_2) - X_{12}} \right\rvert\\
   &= \left\lvert \frac{1}{(1 - (0.0003 - 0.006i)) \cdot (1 - (0.0003 - 0.010i)) - (0.0004 - 0.016i)} \right\rvert\\
   &\approx 1.0006
\end{align*}

\begin{align*}
S_{a,isoliert} &= S_a \cdot VT = 3.702 m/s^2 \cdot 1.0006\\
               &= \underline{\underline{3.704 m/s^2}}
\end{align*}

\pagebreak

\section{Vergleich der Ergebnisse}

\begin{table}[H]
\centering
\begin{tabular}{ |c|c|c| } 
 \hline
 Einmassenschwinger & Vereinfacht & Transmissibilität\\
 \hline\hline
 2.958 $m/s^2$ & 2.858 $m/s^2$ & 3.704 $m/s^2$\\
 \hline
\end{tabular}
\caption{Vergleich der Beschleunigungen aus den drei Ansätzen (Beispiel 1)}
\end{table}

Es zeigt sich, dass jeder Ansatz leicht andere Werte liefert. Die Streuung beträgt $\approx 22\%$. 
Um einen besseren Einblick zu erhalten, sollen die gesamten Spektren betrachtet werden. Dafür wird $k_1$ in Abhängigkeit von der Periode $T$ variiert. Die Spektren sind in \cref{fig:Isolation} dargestellt und zeigen, dass im interessanten Bereich der Perioden niedriger als die des Isolators eine Isolation grundlegend abgebildet werden konnte.

\begin{figure}[H]
    \centering
    \includegraphics[width=1.0\textwidth]{Isolation.png}
    \caption{Isolationsspektren der drei Ansätze (Beispiel 1)}
    \label{fig:Isolation}
\end{figure}

Wie zu erwarten war, ist die Modellierung als effektiver Einmassenschwinger nur eine Näherung. Die Ergebnisse über die Transmissibilität liefern durchweg größere Beschleunigungen als der vereinfachte Ansatz, wobei sich diese Differenz in der Nähe der Isolatorperiode stärker ausprägt. Auch ist anzumerken, dass die Isolationsspektren geglättet sind, da sie aus dem ebenfalls bereits geglätteten Antwortspektrum berechnet wurden.

\section{Vergleich mit den Ergebnissen einer Zeitschrittberechnung}

Um auch einen Vergleich mit den numerisch ermittelten Werten aus \cite{Isemann} zu machen, wurde das Beispiel in Kapitel 11.3 untersucht.
Mit den angegebenen Massen und der Isolatorsteifigkeit stimmt jedoch die Periode des Isolators, die mit $T = 2.251 s$ angegeben wurde, nicht überein.

\begin{align*}
T &= \frac{2 \pi}{\sqrt{(k_2/(m_2+m_1))}}\\
  &= \frac{2 \pi}{\sqrt{(32000/( 2846.7 t + 1619.5 t)}}\\
  &= 2.347 s \neq 2.251 s
\end{align*}

Die Vermutung liegt nahe, dass für $m_1$ die Masse aus dem vorangegangenen Beispiel in den Berechnungen verwendet wurde und es sich in den Angaben um einen \glqq Zahlendreher\grqq{} handelt.

\begin{align*}
T &= \frac{2 \pi}{\sqrt{(32000/( 2486.7 t + 1619.5 t)}}\\
  &= 2.251 s = 2.251 s
\end{align*}

Mit dieser bestätigten Vermutung wird weiterhin mit $m_1 = 2486.7 t$ gerechnet.
Aus der angegebenen Steifigkeit für den Isolator von $k_2 = 32000 kN/m$ lässt sich der verwendete Radius und das Dämpfungsmaß bestimmen.

\begin{align*}
k_2 &= \frac{G}{R} + \mu \frac{G}{D}\\
    &= \frac{41062 kN}{R} + 0.05 \cdot \frac{41062 kN}{0.325 m} = 32000 kN/m \Rightarrow R \approx 1.599 m
\end{align*}

\begin{equation*}
\xi_2 = \xi_{eff} = \frac{2}{\pi} \frac{\mu R}{(D + \mu R)} = \frac{2}{\pi} \frac{0.05 \cdot 1.599 m}{(0.325 m + 0.05 \cdot 1.599 m)} \approx 0.1257
\end{equation*} 

\pagebreak

Damit stehen die Parameter fest und die Spektren können erzeugt werden.

\makebox[1cm]{$D$}    = 0.325 m \par
\makebox[1cm]{$\mu$}  = 0.05\par
\makebox[1cm]{$m_1$}  = 2486.7 t\par
\makebox[1cm]{$m_2$}  = 1619.5 t\par
\makebox[1cm]{$k_2$}  = 32000 kN/m \par
\makebox[1cm]{$R$}    = 1.599 m\par

\begin{figure}[H]
    \centering
    \includegraphics[width=1.0\textwidth]{_Isolation_2.png}
    \caption{Vergleich der Isolationsspektren aus Zeitschrittberechnung \cite{Isemann}, vereinfachtem Ansatz und Ansatz der Transmissibilität (Beispiel 2)}
    \label{fig:Isolation}
\end{figure}

Hier wird deutlich, dass der vereinfachte Ansatz auf der unsicheren Seite liegt.
Bei dem Ansatz der Transmissibilität liegt die Vermutung nahe, dass die Bestimmung der Beschleunigung am gedämpften Antwortspektrum eine doppelte Berücksichtigung der Dämpfung zufolge hatte, da diese bereits in den Transmissionskoeffizienten erfasst wurde.
Eine Anpassung (\cref{fig:Isolation2}) mit $\eta = \sqrt{10/(5+0)} = 1.4142$ zeigt, dass das Isolationsspektrum nun eine Einhüllende darstellt, allerdings auch deutlich zu große Werte ergibt.

\begin{figure}[H]
    \centering
    \includegraphics[width=1.0\textwidth]{_Isolation_3.png}
    \caption{Vergleich der Isolationsspektren aus Zeitschrittberechnung \cite{Isemann}, vereinfachtem Ansatz und Ansatz der Transmissibilität ($\eta = 1.4142$) (Beispiel 2)}
    \label{fig:Isolation2}
\end{figure}

\section{Korrekturansätze}
\label{sec:Korrekturansaetze}

In \cite{Isemann} wurden verschiedene Ansätze untersucht, das Isolationsspektrum nach oben zu korrigieren.
Auch bei dem Ansatz der Transmissibilität könnte man nun versuchen, die Werte nach unten zu korrigieren. Betrachtet man die Transmissionskoeffizienten (\cref{eq:VT2DOF}), so lässt sich erkennen, dass die Dämpfung auch eine Koppelung darstellt. 

Es soll eine grobe Näherung untersucht werden, nach der die Dämpfungsbeiwerte der beiden Systeme getrennt voneinander bestimmt werden.

\begin{align*}
c_1 &= 2 \xi_1 \sqrt{k_1 m_1}\\
c_2 &= 2 \xi_2 \sqrt{k_2 m_2}
\end{align*}

Die Berechnung der Transmissibilität erfolgt dann am nicht transformierten System.

\begin{equation*}
VT(m_2, k_2, c_2, m_1, k_1, c_1)
\end{equation*} 

\begin{table}[H]
\centering
\begin{tabular}{ |c|c|c| } 
 \hline
 Einmassenschwinger & Vereinfacht & Transmissibilität\\
 \hline\hline
 2.958 $m/s^2$ & 2.858 $m/s^2$ & 3.525 $m/s^2$\\
 \hline
\end{tabular}
\caption{Vergleich der Beschleunigungen aus den drei Ansätzen mit Korrektur (Beispiel 1)}
\end{table}

Die Werte aus dem ersten Beispiel liegen nun deutlich näher zusammen.

\begin{figure}[H]
    \centering
    \includegraphics[width=1.0\textwidth]{Isolation_4_2.png}
    \caption{Isolationsspektren der drei Ansätze (Beispiel 1)}
    \label{fig:Isolation2}
\end{figure}


Im zweiten Beispiel (\cref{fig:Isolation21}) kann anhand des Vergleiches der numerisch ermittelten Werte auch festgestellt werden, dass der qualitative Verlauf besser abgebildet werden konnte.

\begin{figure}[H]
    \centering
    \includegraphics[width=1.0\textwidth]{_Isolation_4.png}
    \caption{Vergleich der Isolationsspektren aus Zeitschrittberechnung \cite{Isemann}, vereinfachtem Ansatz und Ansatz der Transmissibilität (Beispiel 2)}
    \label{fig:Isolation21}
\end{figure}

\pagebreak

\chapter{Modalanalyse mit EDV}
\section{Beispielgebäude}
\label{sec:besipielgebaude}

D = 0.325 m
mu = 0.05
m1 = 2486.7 t
m2 = 1619.5 t
k2 = 32000 kN/m
R = 1.599 m

\begin{figure}[H]
    \centering
    \includegraphics[width=1.0\textwidth]{_Isolation_4.png}
    \caption{Vergleich der Isolationsspektren aus Zeitschrittberechnung \cite{Isemann}, vereinfachten Ansatz und Ansatz der Transmissibilität}
    \label{fig:Isolation2}
\end{figure}


\begin{table}[H]
\centering
\begin{tabular}{ |c|c|c| } 
 \hline
 $T [s]$ & $S_a$ Transmissibilität $[m/s^2]$ & $S_a$ Vereinfacht $[m/s^2]$\\
 \hline\hline
0.01 & 5,60674664200148 & 4,67490580045368\\
0.20 & 5,63467499209289 & 4,63042028154013\\
0.40 & 5,71268557687775 & 4,49803343352122\\
0.60 & 5,83030581704598 & 4,28242143699944\\
0.80 & 5,97368100820268 & 3,99218798167146\\
1.00 & 6,12388037700022 & 3,64087769649182\\
1.20 & 6,25648687706686 & 3,24712262982812\\
1.40 & 6,34351458724908 & 2,83331072362479\\
1.60 & 6,35821574645013 & 2,42266417036407\\
1.80 & 6,28168144090241 & 2,03558382801768\\
2.00 & 6,10832152112823 & 1,68670956690896\\
2.20 & 5,84721687995431 & 1,38378025927877\\
2.40 & 5,51863414429968 & 1,12834913010178\\
2.60 & 5,14780403798720 & 0,91758357989544\\
2.80 & 4,75901616490822 & 0,74623440762497\\
3.00 & 4,37194145867057 & 0,60820075617877\\
3.20 & 4,00039726694917 & 0,49752555413495\\
3.40 & 3,65277761818416 & 0,40890321453645\\
3.60 & 3,33323005004162 & 0,33785895564762\\
3.80 & 3,04294312417884 & 0,28074320474709\\
4.00 & 2,78123704406094 & 0,23464010367599\\
 \hline
\end{tabular}
\caption{Spektralbeschleunigungen der Isolationsspektren}
\end{table}


\pagebreak

\section{Berechnung mit RStab}
\label{sec:rstab}

\pagebreak

\section{Diskussion der Ergebnisse}
\label{sec:diskussion}

\pagebreak

\chapter{Analyse}
\section{Diskussion der Ergebnisse}
\label{sec:diskussion}

Es zeigt sich, dass es grundlegend möglich ist nur die aufgehende Struktur zu modellieren und anhand der Isolationsspektren eine Modalanalyse durchzuführen.
Dabei werden die Parameter des Isolators erfasst um ein modifiziertes Antwortspektrum zu erzeugen.
Es wird aber auch deutlich, dass wie bereits erwähnt, der vereinfachte Ansatz für Perioden in der Nähe der Isolatorperiode auf der unsicheren Seite und der Ansatz über die Transmissibilität deutlich auf der sicheren Seite liegt.

\section{Variation des Reibungskoeffizienten $\mu$}
\label{sec:muvariation}

Interessant ist die Beobachtung, dass bei Variation des Reibungskoeffizienten (\cref{fig:muvariation}) deutlich wird, dass die optimale Isolation von $\mu$ abhängig ist und ab einem Grenzwert ein negativer Effekt eintritt.

\begin{figure}[H]
    \centering
    \includegraphics[width=1.0\textwidth]{variation_mu.png}
    \caption{Bild 10.10: Vergleich der Isolationsspektren des „Erdbeben 1“ bei verschiedenen Reibungskoeffizienten \cite{Isemann}}
    \label{fig:muvariation}
\end{figure}

Dieser Effekt kann mit dem vereinfachten Verfahren und dem über die Transmissibilität nicht abgebildet werden. Das liegt daran, dass bei diesen Verfahren $\mu$ einen direkten linearen Einflus auf die effektive Steifigkeit (\cref{keff}) und Dämpfung (\cref{xieff}) hat.

\begin{equation*}
k_{eff} = \frac{G}{R} + \mu \frac{G}{D}
\end{equation*}

\begin{equation*}
\xi_{eff} = \frac{2}{\pi} \frac{\mu R}{(D + \mu R)}
\end{equation*}

Er geht also bei der Linearisierung verloren und kann nur von nicht-linearen Berechnungen, oder empirischen Korrekturfaktoren abgebildet werden.

\pagebreak

\section{Ansätze für verschiende Dämpfungskorrekturbeiwerte $\eta$}

Wie in \cref{sec:Korrekturansaetze} schon erwähnt, wurde in \cite{Isemann} auch eine Korrektur über andere Ansätze für die Bestimmung des Dämpfungskorrekturbeiwertes $\eta$ untersucht.
Der Eurcode 8 sieht dabei folgende Bestimmungsgleichung vor.

\begin{equation*}
\eta = \sqrt{\frac{10}{5+\xi}}
\end{equation*}

Betrachtet man aber die Gleichung für den Vergrößerungsfaktor eines Einmassenschwingers

\begin{equation}
\frac{u_{max}}{u_{stat}} = \frac{1}{\sqrt{(1 - (\omega / \omega_1)^2)^2 + (2 \xi \omega / \omega_1)^2}}
\end{equation}

so ergibt sich im Resonanzfall für $\omega = \omega_1$, nach Umstellung die Beziehung

\begin{equation}
\eta = \frac{1}{2\xi}
\end{equation}

Für $\mu = 0.05$ und $R = 2 m$ zeigt sich aber auch hier, dass die Werte überkorrigiert werden und deutlich zu geringe Beschleunigungen liefern.

\begin{figure}[H]
    \centering
    \includegraphics[width=1.0\textwidth]{Isolation_5.png}
    \caption{}
    \label{fig:Isolation5}
\end{figure}

Nun könnte eine empirische Korrektur vorgenommen werden, die die Ergebnisse aus einer Zeitschrittberechnung mit einfließen lässt. Da diese aber von den Parametern des Bauwerks abhängig sind, wurde an dieser Stelle darauf verzichtet, weil das Ziel sein sollte Isolationsspektren ohne die Erforderniss einer Zeitschrittberechnung zu erstellen.

\pagebreak

\chapter{Zusammenfassung}

In dieser Arbeit wurde ein analytisches Modell erstellt, durch welches isolierte Antwortspektren erzeugt werden können, mit denen es möglich ist Modalanalysen an Gebäudemodellen durchzuführen ohne den Isolator im Gebäudemodell zu erfassen.
Die isolierten Antwortspektren wurden mit dem hier erarbeiteten Modell aus den Antwortspektren gemäß Eurocode 8 abgeleitet.
Dies hat den Vorteil, dass die Antwortspektren nach bereits etablierten Verfahren für den individuellen Gebäudestandort ermittelt werden können.
Das Modell erspart dem Anwender ebenfalls die eventuell komplexe Modellierung des Gebäudeisolators, da das Verhalten des Isolators im isolierten Antwortspektrum abgebildet ist.
Es ist folglich möglich, Verfahren wie die Modalanalyse oder das vereinfachte Antwortspektrenverfahren unter Verwendung des isolierten Antwortspektrums durchzuführen.

Um das Verhalten eines Isolatortyps im Modell abzubilden, wurden in dieser Arbeit ausschließlich Gleitpendelisolatoren betrachtet.
Gute Isolationseigenschaften werden erzielt, wenn die Steifigkeit des Isolators deutlich kleiner ist als die der aufgehenden Struktur und die Masse direkt über dem Isolator möglichst groß ist.
Die Rückstell- und Reibungskraft ist bei Gleitpendelisolatoren proportional zu ihrer effektiven Steifigkeit.
Die Dämpfung wird in linearer Näherung über die Hystereseschleife und die effektive Steifigkeit berechnet.

Um die Ergebnisse des isolierten Antwortspektrums abschätzen zu können, wurden zunächst Näherungen mithilfe des Modells des effektiven Einmassenschwingers berechnet.
Das isolierte Antwortspektrum wurde dabei zunächst mit einem vereinfachten Verfahren berechnet. Dieses Verfahren lieferte jedoch Werte auf der unsicheren Seite, weshalb ein weiterer Ansatz untersucht wurde.
Dieser zweite Ansatz basiert auf der Berechnung der Transmissibilität, um die Übertragung von Schwingungen aus Fußpunktanregungen auf die aufgehende Struktur via des Isolators zu bestimmen.
Bei diesem  Verfahren ist zudem möglich, unterschiedliche Dämpfungswerte für Struktur und Isolator anzunehmen.
Im Verlauf dieser Arbeit wurden daher auch verschiedene Dämpfungsmodelle untersucht.

Die Korrektheit des Ableitungsverfahrens und der Dämpfungsmodelle wurde nachfolgend durch vergleichende Beispielrechnungen verifiziert.
Die Ergebnisse wurden mit den Ergebnissen einer numerischen Zeitschrittberechnung und denen des vereinfachten Verfahrens verglichen.
Bei den Vergleichen zeigte sich, dass das Transmisibilitätsverfahren Werte erzeugt, die auf der sicheren Seite liegen und den qualitativen Verlauf der Spektralbeschleunigungen aus den Zeitschrittberechnungen besser abbilden konnte.

Mit einer Modalanalyse wurde weiterhin gezeigt, dass das isolierte Antwortspektrum dazu verwendet werden kann, das Verhalten des Isolators abzubilden, ohne dass dieser in dem Gebäudemodell vorhanden sein muss.
Es wurden so statische Ersatzlasten berechnet, mithilfe derer das Gebäude auf Erdbebensicherheit ausgelegt werden kann, ohne dass der Isolator aufwendig in das Gebäudemodell integriert werden muss.

\section{Aussicht}

Es hat sich gezeigt, dass mit Korrekturfaktoren die Ergebnisse noch deutlich verbessert werden könnten.

Bei den Isolatorparametern wird die Auslenkung $D$ als die maximale Auslenkung angegeben. Wie sich in \cite{Isemann} aber gezeigt hat, stellt sich diese nicht immer ein. Die tatsächliche Auslenkung ist auch abhängig von der Einwirkung.
Eine Vorgehensweise könnte es nun sein, die Abweichung zwischen maximaler und tatsächlicher Auslenkung über verschiedene Parametersätze und Erdbeben mittels Zeitschrittverfahren zu betrachten und zu untersuchen ob ein Korrekturfaktor abgeleitet werden kann.

Bei der Betrachtung verschiedener Modelle für die Dämpfungskorrektur $\eta$ am Antwortspektrum wurde deutlich, dass auch hier optimiert werden könnte. Ebenfalls wurde in \cite{Isemann} ein realistischerer Dämpfungskorrekturbeiwert als Mittelwert über die Ergebnisse der Zeitschrittberechnung bestimmt. Auch hier wäre es denkbar zu untersuchen, ob es möglich ist, eine Beziehung zu finden, mit der sich der Dämpfungskorrekturbeiwert anpassen ließe.

\section{Excel-Tabelle zur Berechnung von Isolationsspektren}

Die Berechnung des Isolationsspektrums über die Transmissibilität wurde im Rahmen dieser Arbeit mit dem Ansatz aus \cref{sec:Korrekturansaetze} in einer \emph{Excel}-Tabelle implementiert.

So kann nach Eingabe der Parameter 

\makebox[1cm]{$T_B$}  = Eckperiode $[s]$ \par
\makebox[1cm]{$T_C$}  = Eckperiode $[s]$ \par
\makebox[1cm]{$T_D$}  = Eckperiode $[s]$ \par
\makebox[1cm]{$S$}    = Bodenparameter $[-]$ \par
\makebox[1cm]{$a_g$}  = Bodenbeschleunigung $[m/s^2]$ \par

\pagebreak

das Antwortspektrum (für $5 \%$ Dämpfung, $\eta = 1.0$) ermittelt und graphisch dargestellt werden.
Mit der Angabe der beschreibenden Werte des Isolators

\makebox[1cm]{$D$}    = Auslenkung $[m]$ \par
\makebox[1cm]{$R$}    = Radius der Gleitfläche $[m]$ \par
\makebox[1cm]{$\mu$}  = Reibungskoeffizient $[-]$ \par

wird dessen effektive Dämpfung $\xi_2$, Steifigkeit $k_2$ und Periode $T_2$ berechnet. Es fehlen noch die Werte des Bauwerks

\makebox[1cm]{$m_1$}  = Masse der aufgehenden Struktur $[t]$ \par
\makebox[1cm]{$m_2$}  = Masse direkt über dem Isolator $[t]$ \par
\makebox[1cm]{$\xi_1$} = Dämpfung der aufgehenden Struktur $[-]$ \par

und das Isolationsspektrum wird ausgegeben und graphisch dargestellt.

\begin{figure}[H]
    \centering
    \includegraphics[width=1.0\textwidth]{Excel.png}
    \caption{Excel Tabelle zur Berechnung von Isolationsspektren}
    \label{fig:excel}
\end{figure}


\pagebreak

\appendix
%\chapter{Appendix}
\cleardoublepage
\phantomsection
\addcontentsline{toc}{chapter}{Nomenklatur}

\chapter*{Nomenklatur}

\begin{longtable}{cp{3cm}p{8cm}}
\hline
Symbol       & Einheit & Beschreibung \\
\hline\hline
$\omega$     & $1/s$   & Kreisfrequenz\\
$T$          & $s$     & Periode      \\
$t$          & $s$     & Zeit         \\
$a$          & $m/s^2$ & Beschleunigung \\
$S_a$        & $m/s^2$ & Spektralbeschleunigung \\
$F$          & $kN$    & Kraft        \\
$u$          & $m$     & Verschiebung \\
$m$          & $t$     & Masse        \\
$M$          & $t$     & Massenmatrix \\
$k$          & $kN/m$  & Steifigkeit  \\
$K$          & $kN/m$  & Steifigkeitsmatrix \\
$E$          & $kN/m^2$& Elastizitätsmodul \\
$I$          & $m^4$   & Flächenmoment 2. Grades \\
$\xi$        & $-$     & Dämpfungsgrad     \\
$c$          & $\frac{kN}{m/s}$     & Dämpfungskoeffizient \\
$C$          & $\frac{kN}{m/s}$     & Dämpfungsmatrix \\
$\eta$       & $-$     & Dämpfungs-Korrekturbeiwert \\
$P$          & $-$     & Wahrscheinlichkeit \\
$\gamma_1$   & $-$     & Bedeutungsbeiwert \\
$G$          & $t, kN$ & Vertikalkraft (Eigengewicht) \\
$D$          & $m$     & Auslenkung \\
$R$          & $m$     & Radius \\
$\theta$     & $^{\circ}$ & Winkel \\
$\mu$        & $-$     & Reibungskoeffizient \\
$\vec{\Phi}$ & $-$     & Eigenvektor \\
$\varphi$       & $-$     & Eigenvektorkomponente \\
$L$          & $-$     & Beteiligunsfaktor \\
$\vec{I}$    & $-$     & Einheitsvektor \\
$VT$         & $-$     & Transmissibilität \\
\hline
\end{longtable}


\cleardoublepage
\phantomsection
\addcontentsline{toc}{chapter}{\listfigurename}
\listoffigures

\cleardoublepage
\phantomsection
\addcontentsline{toc}{chapter}{Bibliographie}
\printbibliography

\cleardoublepage
\phantomsection
\addcontentsline{toc}{chapter}{Eidesstattliche Erklärung}

\chapter*{Eidesstattliche Erklärung}

Hiermit versichere ich, die vorliegende Master-Thesis ohne Hilfe Dritter, nur mit den
angegebenen Quellen und Hilfsmitteln, angefertigt zu haben. Alle Stellen, die den
Quellen entnommen wurden, sind als solche kenntlich gemacht worden.

\vspace*{\fill}

\begin{tabular}{@{}p{.5in}p{4in}@{}}
& \hrulefill \\
& \GetAuthor \\
& \date{\today{}, Karlsruhe}\\
\end{tabular}

\vspace*{\fill}

\addtocontents{toc}{\cftpagenumbersoff{chapter}}
\addtocontents{toc}{\cftpagenumbersoff{section}}

\cleardoublepage
\addcontentsline{toc}{chapter}{Datenträger}
\addcontentsline{toc}{section}{\texttt{Masterthesis\_Arne\_Rick.pdf}}
\addcontentsline{toc}{section}{\texttt{Isolationsspektrum.xlsx}}
\addcontentsline{toc}{section}{\texttt{Ausdruckprotokoll\_-\_RStab.pdf}}

\addtocontents{toc}{\cftpagenumberson{section}} 
\addtocontents{toc}{\cftpagenumberson{chapter}} 

\cleardoublepage
\phantomsection
\addcontentsline{toc}{chapter}{Berechnungsprotokoll Modalanalyse mit RStab}
\includepdf[pages=-]{Ausdruckprotokoll_-_RStab.pdf}




\end{document}